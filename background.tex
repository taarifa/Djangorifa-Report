\chapter{Background Research}
\label{sec:br}

There are three types of user considered throughout:
\begin{itemize}
\item Super-administrator: the person who configures the system but is not involved in the day-by-day affairs.
\item Administrator: someone at the council who can control and administer the system.
\item Citizen: a person who is interested in reporting problems with a facility.
\end{itemize}

Djangorifa can be divided into two main components:
\begin{itemize}
\item Reports Management
\item Facilities Management
\end{itemize}

The decisions made for the Reports and Facilities Management are discussed in the relevant implementation section Section~\ref{sec:rm}.

A brief introduction to Django and an explanation for why it was chosen can be found in Section~\ref{br:django}.

\section{Django}
\label{br:django}

My requirements for a web framework were:
\begin{itemize}
\item Written in a language I know.
\item Offers geo-spatial database queries.
\item Detailed documentation and a large community of support.
\end{itemize}

PHP was automatically ruled out. I have used it extensively and do not think it is good enough to build a robust, easily maintained codebase. ~\cite{phpsucks} details very clearly the reasons against PHP.

Two frameworks which fit all the requirements were therefore considered: Ruby on Rails~\cite{rails} and Django~\cite{django}.

I decided to try Ruby on Rails. After two days, I realised it does not have object-oriented forms; HTML forms are defined by strings.

Django has object-oriented forms, and these became necessary. Therefore, I switched to Django.

\section{Introduction to Django}
Django is known as an \gls{MVT} framework. Each component of a web application can be separated into an app, which is a folder\footnote{Which must be in the Python path} containing a ``models.py" file, a ``views.py" file and a ``templates" directory. An app may also declare an ``admin.py" file which Django uses to provide an administration section.

The main directory of the web application contains a ``settings.py" file. This file specifies the database in use and the locations of all the apps in the application.

\subsection{Model}
For an app to store data in the database, classes which subclass ``django.db.models.Model" must be defined in ``models.py". These classes are called models. Each model can define class variables called fields. As an example:

\begin{figure}[thp]
	\centering
	\begin{tabular}{c}
	\begin{lstlisting}[label={lst:model}]
	class ExampleModel(models.Model):
		field1 = models.CharField(max_length=2)
		field2 = models.IntegerField(max_digits=2)
		field3 = models.ForeignKey(ExampleModel2)
	\end{lstlisting}
	\end{tabular}
	\caption{Example models}
\end{figure}

The command-line operation ``syncdb" tells Django to create a database table for each class, and each field variable as a column in that table. An automatically generated column called ``id" is also added to the database when this command is called. Django provides an abstraction layer, an \gls{ORM}, so database operations can be performed by calling methods on the models (Figure~\ref{lst:savemodel}).

\begin{figure}[thp]
	\centering
	\begin{tabular}{c}
	\begin{lstlisting}[label={lst:savemodel}]
	# Create a new instance of an example model
	# and save it to the database
	em = ExampleModel(field1="Moo", field2="Blah")
	em.save()
	
	# Get the data from the database and change it
	em = ExampleModel.objects.get(field1="Moo")
	em.field2 = "Something"
	em.save()
	\end{lstlisting}
	\end{tabular}
	\caption{Manipulating models}
\end{figure}

\subsection{View}
When a user navigates to a page of a Django application, Django's URL dispatcher looks for a file called ``urls.py" in the application directory on the server. This file contains all valid URLs for the website, which can be defined either as hard-coded strings or as \glspl{regex}. Each URL is defined with a function to be called when the user requests it. The function is in an app's ``views.py".

Functions in ``views.py" perform the logic needed to get the content to display to the user. These functions are passed a \emph{request} object. This contains \gls{session} information and any data sent with a \gls{get} or a \gls{post}. The view must then return Django's encapsulation of an \gls{HTTP} response.

The HTTP response takes a string which will be sent to the browser as HTML. To maintain Django's philosophy of \gls{DRY}, this string can be rendered using a Django function, ``render", which takes three compulsory arguments: the request object; the path to a template; and a dictionary of variables to give to the template, which can be any Python object, for example a model or a string.

\subsection{Template}
The template is an HTML file inside the ``templates" directory of an app. This file contains a mixture of HTML and Django's template language. The template language provides minimal functionality so that the variables provided in the dictionary can be presented.

\begin{figure}[thp]
	\centering
	\begin{tabular}{c}
	\begin{lstlisting}[label={lst:views}]
return render(request,
	"example.html", {
		'x':'moo',
		'y':[1,2,3]
	}
)
	\end{lstlisting}
	\end{tabular}
	\caption{Views function}
\end{figure}

Figure~\ref{lst:template} is an example of what might be returned by a views function. The template language uses ``\{\{ \}\}" to represent a variable, and ``\{\% \%\}" to represent a macro.

\begin{figure}[thp]
	\centering
	\begin{tabular}{c}
	\begin{lstlisting}[label={lst:template}]
	Cows say {{ x }}.
	\end{lstlisting}
	\end{tabular}
	\caption{Example.html}
\end{figure}

When the template is rendered with the render function, the macros are executed and the variables filled in with the values in the dictionary. When Figure~\ref{lst:template} is rendered, the HTML produced is "Cows say moo.Cows say moo.Cows say moo."

\subsection{Admin}
Django provides an administration section where apps can register their models to be added, changed and deleted through the website. (Figure~\ref{lst:adminmodel})

\begin{figure}[thp]
	\centering
	\begin{tabular}{c}
	\begin{lstlisting}[label={lst:adminmodel}]
	class ExampleModelAdmin(admin.ModelAdmin):
		# Customise the admin inside this class
		pass
		
	# Register with the admin
	admin.site.register(
		ExampleModel,
		ExampleModelAdmin
	)
	\end{lstlisting}
	\end{tabular}
	\caption{Manipulating models}
\end{figure}

\subsection{Forms}
Django provides a forms module which facilitates the creation of HTML forms. django-crispy is a third-party module which makes this process even easier: a form object is rendered in the template by ``\{\% crispy form \%\}".

There are two classes of forms this project uses: ModelForm and Form. ModelForm creates a form for the model it's given.

\begin{figure}[thp]
	\centering
	\begin{tabular}{c}
	\begin{lstlisting}[label={lst:modelform}]
	class ExampleModelForm(forms.ModelForm):
		extra_field = forms.CharField()

		class Meta:
			model = ExampleModel
	\end{lstlisting}
	\end{tabular}
	\caption{ModelForm}
\end{figure}

When Figure~\ref{lst:modelform} is rendered, it produces an HTML form with three input fields: field1, field2 and extra\_field. When this form is submitted, called .save() on the form will save the model with the user input data to the database.

Form provides this ability to generate forms, but with no save method or model.

Both classes are validated by Django's validation method when the form is submitted. Extra validation can be added to the field declaration in the code.

\subsection{Database}
\label{sec:web:database}

Django ships with GeoDjango which provides functionality for geospatial queries on a database. GeoDjango can be used with Postgresql~\cite{postgresql} and MySQL~\cite{mysql}, but the latter has limitations~\cite{mysqllims}, so Postgresql has been used.