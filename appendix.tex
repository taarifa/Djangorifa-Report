\appendix

\chapter{Preventing Corruption}
~\label{app:corruption}

In a continent as rife with population as Africa~\cite{corruption}, it is important to ensure that software is open to as little abuse as possible. The suggested method of doing so is by ensuring that all governmental actions are transparent to the public~\cite{transparency}. This means, for a site such as Taarifa, that an administrator cannot delete any information which the public has submitted.

Spam may be a problem; but transparency is more important.

\chapter{Statuses}
\label{app:statuses}

There are 5 values the status can take: ``Awaiting verification", ``Awaiting assignment", ``Assigned", ``Fixed" and ``Dispute resolution". For each report which enters the system, an administrator has to verify it. This concept is carried over from the Water Hackathon: this was one of the things I implemented on Ushahidi.

The awaiting assignment is awaiting a team to fix the problem. Assigned and fixed are their namesakes. Dispute resolution is when a citizen has reported that their report has not been handled correctly. Once the issue is sorted, the worker will receive their money back, and the report will go back to complete.