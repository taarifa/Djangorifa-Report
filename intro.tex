\chapter{Introduction}
\section{Objective}
The \gls{WBG} is currently funding development projects in Africa. One of the focuses is a centralised reports management system to allow members of the public to report problems with public facilities.

This project aims to create an open-source web application which provides a facilities-reports-management system whereby reports made by citizens can be tracked and managed.

The finished product should be deployable in different countries with minimal configuration, and be easily extensible by other programmers.

\section{Motivation}
This project is motivated by the desire to use the skills gathered during my three years at university to develop a project which is beneficial to people. I was a core member of the winning team at 2011's London Water Hackathon, where I spent 24 hours hacking an existing reports management system, Ushahidi Web~\cite{ushahidiweb}.

This is used primarily in Africa to gather information about currently occurring crises; for example, it was used in the immediate aftermath of the Haitian earthquake.

Ushahidi has many limitation, not least of which that it is being maintained with a deprecated version of Kohana PHP~\cite{kohana}, a web framework. Additionally, it is only applicable for currently occurring problems, there is no follow-up on the reports created.

The civic issue tracker Open311 is used by many cities in the developed world~\cite{open311known} to gather non-urgent reports - FixMyStreet~\cite{fixmystreetuk} is one of the better known implementations - but it is unstable~\cite{open311api}. Once a report has been made, it is sent to the council who are responsible for addressing the problem. If they choose to ignore it, the user does not necessarily receive feedback.

The proposed system aims for a transparent process by facilitating the chain from report-made-on-facility to report-completed. As far as can be discerned, there is no existing product which caters for this.

The target market is councils of developing countries (with the focus on Africa) where there are few, or no, public ICT services. There is therefore a niche for a public facilities manager.

Ushahidi and Open311 are focussed around the web. Over 85\% of internet users in Africa use their mobile phones as a connection point~\cite{africainfo}; although the Internet penetration rate is just 5\%~\cite{africainfo}. Africa also has the fastest growing take-up rate of mobile telephony~\cite{africainfo}.

There is a need for a product which caters to this mobile trend that enables citizens to interact with their local councils.

\subsection{Considerations}
When introduced to the project at the London Water Hackathon, Ushahidi was demonstrated using data from a slum called Tandale in the city of Dar es Saalam, Tanzania. Mark Iliffe~\cite{markiliffe}, the team leader for the project, and others have extensively mapped this area and published the data on OpenStreetMap~\cite{openstreetmap}.

Because there is such extensive data, Tandale will be used as the use-case for the implementation of the project. There is a movement for OpenStreetMap to extend its mapping of Africa~\cite{openstreetmapafrica}, so in the future Tandale will no longer be unusual in this respect.

Bribery and corruption are common in Africa~\cite{corruption}, so any design will be carefully considered to make reporting and the follow-up as transparent as possible to the general public. Preventing nepotism and making the system fair is paramount.

\subsection{Challenges}
A web application has been written in Django - a Python web framework - which has been designed to be pluggable and customisable by a programmer with little skill. The application is named Djangorifa, which is the name of the web application, ``Django", combined with the Swahili for ``reports".

The application consists of a reports management system and a facilities management system.

Performance was a challenge: the technology the application runs off is likely to be old.

This report covers the journey taken in developing the system.